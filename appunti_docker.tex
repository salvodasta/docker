\documentclass[12pt,a4paper]{report}
\usepackage[utf8]{inputenc}
\usepackage{amsmath}
\usepackage{amsfonts}
\usepackage{amssymb}
\usepackage{makeidx}
\usepackage{graphicx}
\usepackage{listings}
\usepackage{mparhack}
\usepackage{hyperref}
%\usepackage[a4paper]{geometry}
\usepackage[left=2.80cm, right=3.80cm, top=2.00cm, bottom=2.00cm]{geometry}
\newcommand{\nota}[2]{%
	\marginpar[{\raggedleft\tiny\sffamily #1\\}]{%
		{\raggedright\tiny\sffamily #1\\}}}
	\marginparwidth = 74pt
	\marginparsep = 21pt
\author{Salvo D'Asta}
\title{Appunti docker}
\makeindex



\begin{document}
	\maketitle
\begin{figure}[h]
	\centering
	\includegraphics[width=0.7\linewidth]{docker-11}
	\caption[docker]{}
\end{figure}

\tableofcontents
\chapter{Cosa è Docker?}
Si tratta di una tecnologia per la distribuzione (\textit{deployment}) di servizi software.
\chapter{differenze tra virtualizzazione e "containerizzazione"}
Il sistema "\textbf{Bare metal}" cioè basato sull'hardware \textit{fisico} della macchina\\
\begin{figure}[h]
	\centering
	\includegraphics[width=0.7\linewidth]{"Schermata a 2021-02-27 08-27-43"}
	\caption{}
	\label{fig:schermata-a-2021-02-27-08-27-43}
\end{figure}
Il sistema basato sulla virtualizzazione della macchina (virtualbox, wmware etc.)
\begin{figure}[p]
	\centering
	\includegraphics[width=0.7\linewidth]{"Schermata a 2021-02-27 08-31-37"}
	\caption{}
	\label{fig:schermata-a-2021-02-27-08-31-37}
\end{figure}\\
\begin{figure}[!]
	\centering
	\includegraphics[width=0.7\linewidth]{"Schermata a 2021-02-27 08-33-02"}
	\caption{}
	\label{fig:schermata-a-2021-02-27-08-33-02}
\end{figure}





  
\chapter{primo uso}	

	\section{Installazione}
	\subsection{Con apt-get install}
	
	\verb|sudo apt-get install docker.io|
	\section{rimuovere i doker file e le immagini}
	\verb|docker system prune|
	\section{creazione ed avvio di un container}
	\verb|docker container run nginx|\\
	Se l'immagine di \verb|ngnix| non si trova in locale, la scarica automaticamente dal \textit{regitry}  
	\section{Elenco dei container attivi e di quelli non attivi}
	\verb|docker container ls| (solo quelli attivi)\\
	\verb|docker container ls -a| (anche quelli inattivi)
	
%	\nota{{{\textbf{Filtro passa basso}:}}}
	
\end{document}
