\documentclass[10pt,a4paper]{report}
\usepackage[utf8]{inputenc}
\usepackage{amsmath}
\usepackage{amsfonts}
\usepackage{amssymb}
\usepackage{makeidx}
\usepackage{graphicx}
\usepackage{listings}
\usepackage{mparhack}
\usepackage{hyperref}
%\usepackage[a4paper]{geometry}
\usepackage[left=0.80cm, right=0.80cm, top=1.80cm, bottom=1.50cm]{geometry}
\newcommand{\nota}[2]{%
	\marginpar[{\raggedleft\tiny\sffamily #1\\}]{%
		{\raggedright\tiny\sffamily #1\\}}}
	\marginparwidth = 74pt
	\marginparsep = 21pt
\author{Salvo D'Asta}
\title{Appunti incompleti su docker}

\makeindex



\begin{document}
	

	
	\maketitle
	
	
%	\begin{abstract}
%		Appunti incompleti su docker. 
%	\end{abstract}


\tableofcontents
\section*{Cosa è Docker?}
Si tratta di una tecnologia per la distribuzione (\textit{deployment}) di servizi software.
differenze tra virtualizzazione e\\ "containerizzazione"
\begin{figure}[h]
	\centering
	\includegraphics[width=0.3\linewidth]{"Schermata a 2021-02-27 08-27-43"}
	\caption{Il sistema "\textbf{Bare metal}" cioè basato sull'hardware \textit{fisico} della macchina}
	\label{fig:schermata-a-2021-02-27-08-27-43}
\end{figure}
\begin{figure}[h]
	\centering
	\includegraphics[width=0.2\linewidth]{"Schermata a 2021-02-27 08-31-37"}
	\caption{Il sistema basato sulla virtualizzazione della macchina (virtualbox, wmware etc.)}
	\label{fig:schermata-a-2021-02-27-08-31-37}
\end{figure}
\begin{figure}[h]
	\centering
	\includegraphics[width=0.2\linewidth]{"Schermata a 2021-02-27 08-33-02"}
	\caption{Il sistema basato su docker}
	\label{fig:schermata-a-2021-02-27-08-33-02}
\end{figure}





\pagebreak
  
\section*{Istruzioni di base}	

	\section{Installazione}
	\subsection{Con apt-get install}
	
	\verb|sudo apt-get install docker.io|
	\section{scaricare un immagine da Docker HUB}
	Collegarsi al sito \textit{https://dockerhub.com} 
	nella sezione \textit{Explore} cercare l'immagine desiderata  e seguire le istruzioni.
	da terminale digitare:\\
	\verb|docker pull <nome immagine>|
	\section{visualizzare l'elenco delle immagini visualizzate}
    \verb|docker images|
	\section{creare ed eseguire un container}
	\verb|docker run -dit --name <nome container>|
	esempio di esecuzione di un server apache:\\
	\verb|docker run -dit --name miocontainer -p 8087:80 -v "/home/pi/classe5D/":/usr/local/apache2/htdocs/ httpd:2.4| 
	\section{visualizzare l'elenco dei container}
	per visualizzare solo i container in esecuzione:\\
	\verb|docker container ls|\\
	per visualizzare anche quelli non attivi:\\
	\verb|docker container ls -a|
	\section{eseguire un comando all'interno di un container}
	\verb|docker exec -it <nome container> <comando>|\\esempio di esecuzione di una shell bash all'interno di un container:\\
	\verb|docker exec -it miohttpd bash | 
	\section{Eliminazione di un container o di una immagine}
	Dopo aver controllato l'elenco dei container:
	\begin{verbatim}
	docker container ls -a
	CONTAINER ID   IMAGE                    COMMAND          CREATED        STATUS       PORTS   NAMES
	d3be21917f8a   portainer/portainer      "/portainer"     3 days ago     Created              lucid_liskov
	6583875c0997   ubuntu                   "/bin/bash"      7 days ago     Exited (0)           mioubuntu
	\end{verbatim}
	per eliminare il container \textit{mioubuntu}:\\
	\verb|docker rm 6583| (non è necessario inserire tutto l'hash)\\
	per eliminare le immagini sostituire il comando \verb|rm| con \verb|rmi| ed inserire i primi numeri dell'hash dell'immagine.
	
	\section{rimuovere i doker file e le immagini}
	\verb|docker system prune|
	\section{creazione ed avvio di un container}
	\verb|docker container run nginx|\\
	Se l'immagine di \verb|ngnix| non si trova in locale, la scarica automaticamente dal \textit{regitry}  
	\section{Elenco dei container attivi e di quelli non attivi}
	\verb|docker container ls| (solo quelli attivi)\\
	\verb|docker container ls -a| (anche quelli inattivi)
	\section{Creazione di una immagine a partire da un container}
	\verb|docker container commit [OPTIONS] CONTAINER [REPOSITORY[:TAG]]|\\
	esempio:\\ 
	\verb|docker container commit <nome del container esistente>  <nome della nuova immagine:<eventuale TAG>>|//
	Supponendo che dopo aver creato il container \verb|mioubuntu| ed aver apportato delle modifiche io voglia creare l'immagine \verb|MIO| allora il comando sarà:\\
	\verb|docker container commit mioubuntu MIO|
	\section{Creazione di un'immagine con DockerFile}
	\begin{figure}[h]
		\centering
		\includegraphics[width=0.3\linewidth]{"Schermata a 2021-02-27 08-33-42"}
		\caption{}
		\label{fig:schermata-a-2021-02-27-08-33-42}
	\end{figure}
	
	\section{creazione di una network}
	da:\verb|docker network  --help|\\

	
	
	\verb|docker network create --subnet=<indirizzo_ip/subnet> <nome della subnet>|
	\section{Esempio creazione container mysql}
	\subsection{creazione del container}
	\verb|docker run --name some-mysql -e MYSQL_ROOT_PASSWORD=my-secret-pw -d mysql:tag|\\
	\subsection{creazione dell'utente remoto}
	\verb|docker exec some_mysql bash|\\
	\verb| mysql -uroot -p|\\
	\verb|CREATE USER 'salvo'@'%' IDENTIFIED BY 'proprietario';|\\
	\verb|GRANT ALL PRIVILEGES ON mysql.* TO 'salvo'@'%';|\\
	
%	\nota{{{\textbf{Filtro passa basso}:}}}
\section{Installazione su Raspberry Pi con Moodlebox}
\verb|ssh moodlebox@moodlebox.home|	\\
\verb|sudo apt-get install docker.io|\\
\verb|sudo apt-get install -y --no-install-recommends \|\\
\verb|apt-transport-https \|\\
\verb|ca-certificates \|\\
\verb|curl \|\\
\verb|software-properties-common|\\\\
\verb|curl -fsSL https://apt.dockerproject.org/gpg | sudo apt-key add -|
\\\\
root@raspberrypi:/home/pi\# docker images\\
	REPOSITORY          TAG       IMAGE ID       CREATED       SIZE\\
	httpd               latest    d8c0ac39cbcb   2 weeks ago   107MB\\
	armhf/hello-world   latest    d40384c3f861   4 years ago   1.64kB\\
	\\\\
	 docker run -itd --name mio-httpd:latest -p 8080:80 httpd:latest
	\\\\
	root@raspberrypi:/home/pi\# docker exec -it mio-httpd  bash\\
	root@f2ca72141b9e:/usr/local/apache2\#
	\\
	\\
	cat>index.html\\
	<h1> Container httpd in esecuzione su raspberry pi4 4GB <\/h1>\\
	\^Z
	[1]+  Stopped                 cat > index.html\\
	 
	

\end{document}
